\documentclass[a4paper,UTF8,11pt]{article}
\usepackage{xeCJK}

%\usepackage{ctex}
\usepackage[margin=1.25in]{geometry}
\usepackage{color}
\usepackage{graphicx}
\usepackage{amssymb}
\usepackage{amsmath}
\usepackage{amsthm}
\usepackage{enumerate}
\usepackage{bm}
\usepackage{hyperref}
\usepackage{epsfig}
\usepackage{indentfirst}
\usepackage{color}
\usepackage{mdframed}
\usepackage{lipsum}
\usepackage{mathtools}
\usepackage{algorithm}
\usepackage{algorithmic}
\usepackage{booktabs}
\usepackage{longtable}

\newmdtheoremenv{thm-box}{myThm}
\newmdtheoremenv{prop-box}{Proposition}
\newmdtheoremenv{def-box}{define}

\hypersetup{hidelinks}

\setlength{\evensidemargin}{.25in}
\setlength{\textwidth}{6in}
\setlength{\topmargin}{-0.5in}
\setlength{\topmargin}{-0.5in}
% \setlength{\textheight}{9.5in}
%%%%%%%%%%%%%%%%%%set header and footer here%%%%%%%%%%%%%%%%%%
\usepackage{fancyhdr}                                
\usepackage{lastpage}                                           
\usepackage{layout}                                             
\footskip = 10pt 
\pagestyle{fancy}                    
\lhead{2019, Spring}                    
\chead{Computer Vision: Representation and Recognition}
\rhead{Assignment 2}                                                                                               
\cfoot{\thepage}                                                
\renewcommand{\headrulewidth}{1pt}  			%header
\setlength{\skip\footins}{0.5cm}    			
\renewcommand{\footrulewidth}{0pt}  
	

\makeatletter 							
\def\headrule{{\if@fancyplain\let\headrulewidth\plainheadrulewidth\fi%
\hrule\@height 1.0pt \@width\headwidth\vskip1pt	
\hrule\@height 0.5pt\@width\headwidth  			
\vskip-2\headrulewidth\vskip-1pt}      			
 \vspace{6mm}}     						
\makeatother  

%%%%%%%%%%%%%%%%%%%%%%%%%%%%%%%%%%%%%%%%%%%%%%
\numberwithin{equation}{section}
\newtheorem{myThm}{myThm}
\newtheorem*{myDef}{Definition}
\newtheorem*{mySol}{Solution}
\newtheorem*{myProof}{Proof}
\newcommand{\indep}{\rotatebox[origin=c]{90}{$\models$}}
\newcommand*\diff{\mathop{}\!\mathrm{d}}

\usepackage{multirow}
\renewcommand\refname{reference}

\linespread{1.5}
\setlength{\parskip}{4pt}

\setlength{\parindent}{2em}	
%%%%%%%%%%%%%%%%%%%%%%%%%%%%%%%%%%%%%%%%%%%%%%%%%%%%%%%%%%%%%%%%

\begin{document}
\title{Computer Vision: Representation and Recognition\\
Assignment 2}
\author{161180038, 广进, \href{mailto:guangjin1998@gmail.com}{guangjin1998@gmail.com}}
\maketitle

\section{Canny Edge Detector (30 points)}
\subsection{Will the rotated edge be detected using the same Canny edge detector?}
假设之前的点为$(x,y)$,对应函数为$ f(x,y) $,旋转之后点为$(x',y')$,对应函数为$ g(x',y') = f(x,y) $,经过旋转有如下关系:
\begin{equation}
{
\left[ \begin{array}{c}
x'\\
y'
\end{array}
\right ]}
=
{
\left[ \begin{array}{cc}
cos(\theta) & -sin(\theta)\\
sin(\theta) & cos(\theta)
\end{array} 
\right ]}
{
\left[ \begin{array}{c}
x\\
y
\end{array}
\right ]}
\Rightarrow
{
\left[ \begin{array}{c}
x\\
y
\end{array}
\right ]}
=
{
\left[ \begin{array}{cc}
cos(\theta) & sin(\theta)\\
-sin(\theta) & cos(\theta)
\end{array} 
\right ]}
{
\left[ \begin{array}{c}
x'\\
y'
\end{array}
\right ]}
\end{equation}

则有,
\begin{align*}
\dfrac{\partial g(x',y')}{\partial x'} &= \dfrac{\partial f(x,y)}{\partial x'} \\
&=\dfrac{\partial f(x,y)}{\partial x}\dfrac{\partial x}{\partial x'}+\dfrac{\partial f(x,y)}{\partial y}\dfrac{\partial y}{\partial x'}\\
&=\dfrac{\partial f(x,y)}{\partial x} cos(\theta)+\dfrac{\partial f(x,y)}{\partial y}(-sin(\theta))
\end{align*}
\begin{align*}
\dfrac{\partial g(x',y')}{\partial y'} &= \dfrac{\partial f(x,y)}{\partial y'} \\
&=\dfrac{\partial f(x,y)}{\partial x}\dfrac{\partial x}{\partial y'}+\dfrac{\partial f(x,y)}{\partial y}\dfrac{\partial y}{\partial y'}\\
&=\dfrac{\partial f(x,y)}{\partial x} sin(\theta)+\dfrac{\partial f(x,y)}{\partial y}cos(\theta)
\end{align*}

从而有,
\begin{align*}
(\dfrac{\partial g(x',y')}{\partial x'})^2&=(\dfrac{\partial f(x,y)}{\partial x} cos(\theta)+\dfrac{\partial f(x,y)}{\partial y}(-sin(\theta)))^2\\
&=(\dfrac{\partial f(x,y)}{\partial x} cos(\theta))^2+(\dfrac{\partial f(x,y)}{\partial y}sin(\theta))^2-2\dfrac{\partial f(x,y)}{\partial x}\dfrac{\partial f(x,y)}{\partial y}sin(\theta)cos(\theta)
\end{align*}
\begin{align*}
(\dfrac{\partial g(x',y')}{\partial y'})^2&=(\dfrac{\partial f(x,y)}{\partial x} sin(\theta)+\dfrac{\partial f(x,y)}{\partial y}cos(\theta))^2\\
&=(\dfrac{\partial f(x,y)}{\partial x} sin(\theta))^2+(\dfrac{\partial f(x,y)}{\partial y}cos(\theta))^2+2\dfrac{\partial f(x,y)}{\partial x}\dfrac{\partial f(x,y)}{\partial y}sin(\theta)cos(\theta)
\end{align*}

故有 the magnitude of its derivative:
\begin{equation}
\sqrt{(\dfrac{\partial g(x',y')}{\partial x'})^2+(\dfrac{\partial g(x',y')}{\partial y'})^2}
=\sqrt{(\dfrac{\partial f(x,y)}{\partial x})^2+(\dfrac{\partial f(x,y)}{\partial y})^2}
\end{equation}

Therefore, the rotated edge will be detected using the same Canny edge detector.

\subsection{how to adjust the threshold (up or down) to address both problems}
Canny算法中减少假边缘数量的方法是采用双阈值法\footnote{Canny边缘检测算法原理及其VC实现详解(一) https://blog.csdn.net/likezhaobin/article/details/6892176 \\}。选择两个阈值,根据高阈值得到一个边缘图像,这样一个图像含有很少的假边缘,但是由于阈值较高,产生的图像边缘可能不闭合,未解决这样一个问题采用了另外一个低阈值。在高阈值图像中把边缘链接成轮廓,当到达轮廓的端点时,该算法会在断点的8邻域点中寻找满足低阈值的点,再根据此点收集新的边缘,直到整个图像边缘闭合。

Long edges are broken into short segments separated by gaps: 是因为介于高阈值和低阈值中间没有足够的候选者,无法产生闭合边。所以应该将低阈值降低以有更多候选者。

Some spurious edges appear: 是因为假边有一部分误以为是必须要的,应该通过提高高阈值来抑制假边。

最简单方法\footnote{Canny Edge Detection Auto Thresholding http://www.kerrywong.com/2009/05/07/canny-edge-detection-auto-thresholding/}:使用平均值或者中位数。令 high threshold 为1.33倍的平均值/中位数,low threshold 为0.67倍的平均值/中位数




\newpage
\section{Difference-of-Gaussian (DoG) Detector (30 points)}
本部分代码请见 DoG.ipynb,不过图示部分均在pdf有展示。
\subsection{2nd derivative with respect to x}
The 1-D Gaussian is
\begin{equation*}
g_{sigma}(x)=\dfrac{1}{\sqrt{2\pi}\sigma}exp(-\dfrac{x^2}{2\sigma^2})
\end{equation*}

1st derivative with respect to x is
\begin{align*}
g_{sigma}'(x)&=\dfrac{1}{\sqrt{2\pi}\sigma}exp(-\dfrac{x^2}{2\sigma^2})*(-\dfrac{x}{\sigma^2})\\
&=-\dfrac{x}{\sqrt{2\pi}\sigma^3}exp(-\dfrac{x^2}{2\sigma^2})
\end{align*}

2nd derivative with respect to x is
\begin{align*}
g_{sigma}''(x)&=-\dfrac{1}{\sqrt{2\pi}\sigma^3}exp(-\dfrac{x^2}{2\sigma^2})-\dfrac{x}{\sqrt{2\pi}\sigma^3}exp(-\dfrac{x^2}{2\sigma^2})*(-\dfrac{x}{\sigma^2})\\
&=\dfrac{1}{\sqrt{2\pi}\sigma^3}(\dfrac{x^2}{\sigma^2}-1)exp(-\dfrac{x^2}{2\sigma^2})
\end{align*}

use Python to plot it (use σ = 1)
	\begin{figure}[h]
		\centering  %centering image
		\includegraphics[scale=0.7]{Gauss2.png}  % load image
		\caption{2nd derivative with respect to x }  %
	\end{figure}
	
	
\subsection{plot the difference of Gaussians in 1-D}	
Use Python to plot them (use σ = 1, k = 1.2, 1.4, 1.6, 1.8, 2.0), and k = 1.2 gives the best approximation to the 2nd derivative with respect to x.
\begin{figure}[h]
\centering
\begin{minipage}[t]{0.48\textwidth}
\centering
\includegraphics[width=7.5cm]{1-2.png}
\end{minipage}
\begin{minipage}[t]{0.48\textwidth}
\centering
\includegraphics[width=7.5cm]{1-4.png}
\end{minipage}

\centering
\begin{minipage}[t]{0.48\textwidth}
\centering
\includegraphics[width=7.5cm]{1-6.png}
\end{minipage}
\begin{minipage}[t]{0.48\textwidth}
\centering
\includegraphics[width=7.5cm]{1-8.png}
\end{minipage}

\centering
\begin{minipage}[t]{0.48\textwidth}
\centering
\includegraphics[width=7.5cm]{2-0.png}
\end{minipage}
\begin{minipage}[t]{0.48\textwidth}
\centering
\includegraphics[width=7.5cm]{1-001.png}
\end{minipage}
\end{figure}

Morever, We can see that 1st derivative with respect to $\sigma $ is
\begin{align*}
\dfrac{\partial g_{sigma}}{\partial \sigma} &= \dfrac{1}{\sqrt{2\pi}}(-\dfrac{1}{\sigma^2})exp(-\dfrac{x^2}{2\sigma^2})+\dfrac{1}{\sqrt{2\pi}\sigma}exp(-\dfrac{x^2}{2\sigma^2})(-\dfrac{x^2}{2})(-2\dfrac{1}{\sigma^3}))\\
&=\dfrac{1}{\sqrt{2\pi}\sigma^2}(\dfrac{x^2}{\sigma^2}-1)exp(-\dfrac{x^2}{2\sigma^2})\\
&=\sigma  \dfrac{\partial^2 g_{sigma}}{\partial^2 x}  
\end{align*}

When $\sigma = 1$, $\dfrac{\partial g_{sigma}}{\partial \sigma}=\dfrac{\partial^2 g_{sigma}}{\partial^2 x}  $ , so $k\rightarrow 1 $ gives the best approximation to the 2nd derivative with respect to x. And we can see k=1.001 is better than k=1.2

\subsection{The 2D equivalents of the plots above are rotationally symmetric. To what type of image structure will a difference of Gaussian respond maximally?}
由下图\footnote{图片来自于https://blog.csdn.net/pi9nc/article/details/18619893}结合上面DoG图像,可以看到,DoG对中心点负响应最大,周围有一圈正响应。所以如果做卷积,对于黑点(周围白背景且点的范围也要合适)响应最大。
	\begin{figure}[h]
		\centering  %centering image
		\includegraphics[height=8cm]{2D.png}  % load image
		\caption{2D高斯二阶导}  %
	\end{figure}
	
另外,使用DoG算子对图像做处理,其极大值和极小值还可以检测角点。

\newpage


\section{Edge detector(40 points)}
本部分代码请见 Edge\_detector.py, 详细使用方法请见 README.md

\subsection{效果展示}
\subsubsection{Lenna  threshold:0.015}
	\begin{figure}[h]
		\centering  %centering image
		\includegraphics[width=13cm]{0_015_Lenna.png}  % load image
		\caption{Lenna Edge}  %
	\end{figure}
\newpage
\subsubsection{Cup  threshold:0.5}
图片来自于昵图网\footnote{http://pica.nipic.com/2007-11-26/200711262323153\_2.jpg}
\begin{figure}[h]
\centering
\begin{minipage}[t]{0.48\textwidth}
\centering
\includegraphics[width=7.5cm]{200711262323153_2.jpg}
\end{minipage}
\begin{minipage}[t]{0.48\textwidth}
\centering
\includegraphics[width=7.5cm]{323153_2.jpg}
\end{minipage}
\end{figure}

\subsubsection{Blueberry and Cup  threshold:8}
图片来自于http://www.weimeiba.com\footnote{http://old.bz55.com/uploads/allimg/140903/138-140Z3093610.jpg}
\begin{figure}[h]
\centering
\begin{minipage}[t]{0.48\textwidth}
\centering
\includegraphics[width=7.5cm]{138-140Z3093610.jpg}
\end{minipage}
\begin{minipage}[t]{0.48\textwidth}
\centering
\includegraphics[width=7.5cm]{8_138-140Z3093610.jpg}
\end{minipage}
\end{figure}

\newpage
\subsubsection{Pandas  threshold:8}
图片来自于互动百科、昵图网\footnote{http://a4.att.hudong.com/63/06/16300000291746124581064816436.jpg}
\begin{figure}[h]
\centering
\begin{minipage}[t]{0.48\textwidth}
\centering
\includegraphics[width=7.5cm]{8816813_225638566000_2.jpg}
\end{minipage}
\begin{minipage}[t]{0.48\textwidth}
\centering
\includegraphics[width=7.5cm]{8_8816813_225638566000_2.jpg}
\end{minipage}
\end{figure}

\subsubsection{Teapot  threshold:15}
茶壶茶杯图片来自于昵图网\footnote{http://pic20.nipic.com/20120427/3177520\_175320712116\_2.jpg}
\begin{figure}[h]
\centering
\begin{minipage}[t]{0.48\textwidth}
\centering
\includegraphics[width=7.5cm]{th.jpeg}
\end{minipage}
\begin{minipage}[t]{0.48\textwidth}
\centering
\includegraphics[width=7.5cm]{15_th.jpeg}
\end{minipage}
\end{figure}
\end{document}










\end{document}